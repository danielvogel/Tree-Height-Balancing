\chapter{Fazit}
\ac{THB} ist eine einfache Möglichkeit die Performance von Programmen zu erhöhen. Insbesondere in Anbetracht der Tatsache, dass durch das Fehlen eines Kontrollflusses im Basisblock wenig Analyse nötig ist, ist \ac{THB} bei Multiprozessor-Systemen von großem Nutzen. Andere Systeme mit nur einem Kern bzw. Prozessor können durch diese Optimierungen nicht profitieren. Dies ist der parallelen Ausführung von Operationen geschuldet. Jede Operation muss auf einem eigenen Prozessor/Kern durchgeführt werden. Da heute allerdings meistens Multiprozessor-Systeme genutzt werden, ist die Optimierung eines Programms mit Hilfe des \ac{THB} Algorithmus sehr sinnvoll. Die Kombination mit weiteren Optimierungen bietet die Möglichkeit die Performance noch weiter zu steigern.

Weiterhin haben die Tests, welche zur Kontrolle des Algorithmus durchgeführt wurden, gezeigt, dass er funktioniert und einsetzbar ist. Aktuelle Compiler nutzen ihn schon zur Optimierung. Mit dem Trend zu immer mehr Kernen bzw. Prozessoren wird ein Einsatz dieses Algorithmus auch immer sinnvoller. Programme werden zunehmend komplexer und benötigen immer mehr Rechenleistung. Da die Taktraten mittlerweile an ihre Grenzen geraten, ist nur durch eine Parallelisierung von Programmabläufen eine Verbesserung der Ausführungszeit möglich. Der Einsatz von Optimierungen die an dieser Stelle ansetzen wird somit immer wichtiger.