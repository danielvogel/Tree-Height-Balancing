\documentclass[12pt,a4paper,titlepage,oneside]{scrreprt}

% Schriftart
\usepackage[utf8]{inputenc}
\usepackage[german]{babel}
%\usepackage[T1]{fontenc}
%\usepackage[scaled]{uarial} %WINDOWS only!!!
%\renewcommand*\familydefault{\sfdefault}
%\sloppy


% Meta Information festlegen
\usepackage[
pdftitle={Ausarbeitung für das Modul Compilerbau 2},
pdfsubject={Tree-Height Balancing},
pdfauthor={Sören Gutzeit, Patrick Koteras, Artjom Siebert, Daniel Vogel},
pdfkeywords={Compilerbau, Tree, Height, Balancing}
]{hyperref}

%Bilder
\usepackage[all]{hypcap}
\usepackage{graphicx}
\usepackage{float}
\usepackage{scrhack}

% Deutsches Datumsformat
\usepackage{datetime}
\newdateformat{mydate}{\twodigit{\THEDAY}.\twodigit{\THEMONTH}.\THEYEAR}
\mydate

% Acronyme für Glossar
\usepackage[printonlyused]{acronym}

% Listings
\usepackage{listings}

\usepackage{color}
\definecolor{mygreen}{rgb}{0,0.6,0}
\definecolor{mygray}{rgb}{0.5,0.5,0.5}
\definecolor{mymauve}{rgb}{0.58,0,0.82}
\lstset{ %
  backgroundcolor=\color{white},   % choose the background color; you must add \usepackage{color} or \usepackage{xcolor}
  basicstyle=\footnotesize,        % the size of the fonts that are used for the code
  breakatwhitespace=false,         % sets if automatic breaks should only happen at whitespace
  breaklines=true,                 % sets automatic line breaking
  captionpos=b,                    % sets the caption-position to bottom
  commentstyle=\color{mygreen},    % comment style
  deletekeywords={...},            % if you want to delete keywords from the given language
  escapeinside={\%*}{*)},          % if you want to add LaTeX within your code
  extendedchars=true,              % lets you use non-ASCII characters; for 8-bits encodings only, does not work with UTF-8
  frame=single,                    % adds a frame around the code
  keepspaces=true,                 % keeps spaces in text, useful for keeping indentation of code (possibly needs columns=flexible)
  keywordstyle=\color{blue},       % keyword style
  language=C,                 % the language of the code
  morekeywords={*,...},            % if you want to add more keywords to the set
  numbers=left,                    % where to put the line-numbers; possible values are (none, left, right)
  numbersep=5pt,                   % how far the line-numbers are from the code
  numberstyle=\tiny\color{mygray}, % the style that is used for the line-numbers
  rulecolor=\color{black},         % if not set, the frame-color may be changed on line-breaks within not-black text (e.g. comments (green here))
  showspaces=false,                % show spaces everywhere adding particular underscores; it overrides 'showstringspaces'
  showstringspaces=false,          % underline spaces within strings only
  showtabs=false,                  % show tabs within strings adding particular underscores
  stepnumber=1,                    % the step between two line-numbers. If it's 1, each line will be numbered
  stringstyle=\color{mymauve},     % string literal style
  tabsize=2,                       % sets default tabsize to 2 spaces
  title=\lstname                   % show the filename of files included with \lstinputlisting; also try caption instead of title
}

% biblatex einfügen
\usepackage[babel]{csquotes}
\usepackage[style=alphabetic,hyperref=true,backend=bibtex8]{biblatex}
\bibliography{literatur}

% Links richtig färben
\hypersetup{
	linkcolor		= black,	% red 		Color for normal internal links. 
	anchorcolor		= black,	% black 	Color for anchor text. 
	citecolor		= black,	% green 	Color for bibliographical citations in text. 
	filecolor		= black,	% magenta	Color for URLs which open local files. 
	menucolor		= black,	% red 		Color for Acrobat menu items. 
	urlcolor		= black,	% cyan 		Color for linked URLs.
	colorlinks		= true		% Use colored text instead of a frame around it.
}

% Eigene Befehle
\renewcommand{\baselinestretch}{1.50}\normalsize
\newcommand{\acfoot}[1]{\acs{#1}\footnote{\acl{#1}}}

\author{Sören Gutzeit, Patrick Koteras, Artjom Siebert, Daniel Vogel}
\title{Tree-Height Balancing}

% Titlepage
\begin{document}
\begin{titlepage}
	\begin{center}
		\includegraphics[width=0.5\textwidth]{images/logo.png}\\
		\vspace{3cm}
		\begin{Huge}
			 Tree-Height Balancing
		\end{Huge} \linebreak
		Ausarbeitung im Modul Compilerbau 2 \\
		bei Herrn Jens Mehler (Wintersemster 2014/15)\\
		\vspace{3cm}
		Sören Gutzeit, Patrick Koteras, Artjom Siebert und Daniel Vogel \\
		\today
	\end{center}
\end{titlepage}

% Leere Seite nach Titlepage
\newpage 
\thispagestyle{empty}
\quad
\newpage

% Inhaltsverzeichnis
\renewcommand{\baselinestretch}{1.30}\normalsize
\pagenumbering{Roman}
\setcounter{page}{1}
\tableofcontents
\renewcommand{\baselinestretch}{1.50}\normalsize

% Sichern der aktuellen Seitenzahl des Vorspanns
\newcounter{last_roman}
\setcounter{last_roman}{\value{page}}
\newpage

%Seiten
\pagenumbering{arabic}
\chapter{Einführung}
\label{Einfuehrung}

\section{Nutzen und Erwartungen}
\label{Nutzen}

Lineare Baumstrukturen entstehen oft durch lineare Blockabhängigkeiten. Diese Abhängigkeiten sind jedoch oft nicht zwingend linear, sonder werden der Einfachheit halber vom Compiler so dargestellt. Beispiele hierfür finden sich oft in verschachtelten Strukturen von algebraischen Relationen. \\
Die folgende Rechnung \ref{eq:beispiel-addition} wird oft mit einem rechts- oder links-assoziativen Abhängigkeitsgraph abgebildet. Dies führt dazu, dass in Mehrkernsystem die einzelnen Operationen nicht parallel durchgeführt werden können.

\begin{equation} \label{eq:beispiel-addition}
a + b + c + d + e + f + g + h
\end{equation}

In einer links-assoziativen Baumstruktur (siehe Abbildung \ref{fig:links-assoziativer-baum}) ist ein linearer Programmfluss vorgegeben. Jeder Schritt baut hierbei auf die vorherige Operation auf. In Tabelle \ref{tab:links-assoziativer-baum} sind die Befehle aufgelistet, welche aus dem links-assoziativen Baum in Abbildung \ref{fig:links-assoziativer-baum} folgen. Die Befehle sollen hierbei für einem 2-Kern-System optimiert werden. Die Bezeichner \textit{t1} bis \textit{t7} stehen dabei für die einzelnen Teilbäume, beziehungsweise ihre Zwischenergebnisse. Wie zu sehen werden die Befehle nur auf einen Kern ausgeführt. Der andere Kern kann nicht agieren, da ihm immer eine Abhängigkeit zum Folgebefehl fehlt.\\

Wünschenswert wäre an dieser Stelle jedoch ein Befehlssatz, welcher auf Mehrkernsystemen (zum Teil) parallel ausgeführt werden kann. Dadurch könnten Prozessortakte und somit die Laufzeit des kompilierten Programmes eingespart werden.\\
Das Verfahren des Tree-Height-Balancing soll hierbei angewendet werden, um die links- oder rechts-assoziativen Bäume auszubalancieren. Sofern die Kinder eines Baumes nicht untereinander Abhängigkeiten aufweisen, können diese parallel vom Prozessor bearbeitet werden. Die führt zur Ausführung vom mehreren Befehlen innerhalb eines Taktes in Mehrkernsystemen.


\begin{figure}
	\begin{center}
	\includegraphics[width=0.5\textwidth]{images/links_assoziativer_baum}\\
	\end{center}
	\caption{Links-assoziativer Baum}
	\label{fig:links-assoziativer-baum}
\end{figure}

\begin{table}
	\begin{center}
			\begin{tabular}{|c|c|c|}
				\hline  & Kern 1 & Kern 2 \\ 
				\hline 1 & $ t1 \leftarrow a + b $& - \\ 
				\hline 2 & $ t2 \leftarrow t1 + c $& - \\ 
				\hline 3 & $ t3 \leftarrow t2 + d $& - \\ 
				\hline 4 & $ t4 \leftarrow t3 + e $& - \\ 
				\hline 5 & $ t5 \leftarrow t4 + f $& - \\ 
				\hline 6 & $ t6 \leftarrow t5 + g $& - \\ 
				\hline 7 & $ t7 \leftarrow t6 + h $& - \\ 
				\hline 
			\end{tabular}
	\end{center}
	\caption{Befehle für links-assoziativen Baum}
	\label{tab:links-assoziativer-baum}
\end{table}

\newpage
\section{Dependency Graph als Einstiegspunkt}
\begin{quotation}
"'Compilers also use graphs to encode the flow of values from the point where a value is created, a definition, to any point where it is used, a use. A data-dependency graph embodies this relationship." \cite{HeBIS-309344573}
\end{quotation}
Ein Abhängigkeitsgraph (dependency graph) stellt den Informationsfluss zwischen den Attributinstanzen eines bestimmten Parse-Baumes dar. Eine Kante von einer Attributinstanz zu einer anderen bedeutet, dass der Wert der ersten benötigt wird, um den der zweiten zu brechnen. Kanten drücken die durch die semantische Regeln auferlegten Einschrängungen aus. "'Für jeden \textbf{Knoten} des Parse-Baumes, der mit dem Grammatiksymbol X bezeichnet sei, weist der Abhängigkeitsgraph für jedes mit X verbundene Attribut einen Knoten auf. Angenommen, eine mit einer Produktion p verbundene semantische Regel definiert den Wert des synthetisierten Attributes A.b durch den Wert X.c . Dann hat der Abhängigkeitsgraph eine \textbf{Kante} von X.c nach A.b . \cite{HeBIS-194410269}"' 

Der Algorithmus verlangt einen Abhängigkeitsgraphen, der in der ersten Phase bearbeitet wird. Zur Erstellung eines Abhängigkeitsgraphen bedarf es den Vorstufen des Compilers. Um es kurz zu halten: Ein Scanner muss die Tokens mit Hilfe von regulären Ausdrücken einlesen und ein Parser diese in einen Parse-Baum wandeln. Aufgrund zeitlicher Begrenzung und der durch den Algorithmus gegebenen Komplexität, wird ein Abhängigkeitsgraph manuel erstellt. Dieser Ablauf wird im folgenden näher erklärt.

\subsection{Graph - Linked Implementation} 
Ein Graph ist eine Datenstruktur, in der Informationen gespeichert werden können. Im Gegensatz zu Bäumen, die eine hierarchische Struktur besitzen, sind Graphen flexibler. Die Konsequenz daraus ist, dass Graphen auch Schleifen haben können und ein Knoten in einem Graphen nicht unbedingt erreichbar sein muss (siehe Abbildung \ref{fig:Graph}).
\begin{figure}[h]
\centering
\includegraphics[scale=0.5]{images/Graph.png} 
\caption{Beispiel eines Graphen.}
\label{fig:Graph}
\end{figure}
Wie in Abbildung \ref{fig:Graph} zu sehen ist, besteht der Graph aus den Knoten A, B, C, D, E und F. Diese Knoten werden auch \textit{vertices} oder auch \textit{vertex} bezeichnet. Knoten können, müssen aber nicht, mit einander verbunden sein. Eine Verbindung zwischen zwei Knoten wird als Kante oder auch \textit{edge} definiert und wird grafisch durch die gerichteten Pfeile dargestellt. Die Informationen sind in den jeweiligen Knoten gespeichert und werden durch die Kanten in Abhängigkeit gesetzt. 

\subsubsection*{Darstellung in C}
Ein Graph wird oft mit Hilfe einer \textit{Adjazenzmatrix\footnote{Wird auch Nachbarschaftsmatrix genannt}} oder einer \textit{Adjazenzliste} abgebildet. Eine Adjazenzmatrix ist eine 2D N x N Matrix, wobei N die Einträge der Knoten des Graphen sind. Die Matrix wird so aufgebaut, dass eine Verbindung zwischen zwei Knoten mit einer \textbf{1} versehen werden. Alle anderen Felder sind mit einer \textbf{0} gekennzeichnet. Dies hat Zufolge, dass viel Speicherplatz dadurch verbraucht wird, indem die unrelevante Information (keine Verbindung) ebenfalls dargestellt wird. Des Weiteren muss die Größe des Arrays bekannt sein. 

Abhilfe schafft hier die Adjazenzliste. Diese Liste kann mit einer "'Linked List"' in C abgebildet werden. Dabei spielt die Anzahl der Knoten keine Rolle und wird nur durch den pyhsikalischen Speicher begrenzt. Abbildung \ref{fig:adjacencelist} zeigt den Graphen in einer Adjazenzliste. Ein Eintrag in der vertikalen Liste beinhaltet jeweils einen Knoten des Graphen und hat ein Verweis auf deren Nachfolger. Jeder Knoten speichert seine Verbindungen, also Kanten, ab. 
\begin{figure}[h]
\centering
\includegraphics[scale=0.85]{images/adjacencelist.pdf} 
\caption{Adjazenzliste des Beispiel-Graphen.}
\label{fig:adjacencelist}
\end{figure}
Eine Implementation einer Adjazenzliste ist im Listing \ref{list:graph} abgelegt. Die vertikale Liste ist die \textit{vertex - Liste} und die horizontale Liste ist die \textit{edege - Liste}. Das Element \textit{connectTo} ist ein Pointer auf ein Vertex. Dadurch werden Kanten zwischen den Knoten definiert. Es wird also kein neuer Knoten angelegt, sondern auf einen der im Graphen enthalten ist, verwiesen. Ein Abhängigkeitsgraph speichert einen Graphen und eine Liste, der im Graphen befindlichen Variablen (UEVARS), ab.
\begin{lstlisting}[caption=Struktur eines Graphen., label=list:graph]
typedef struct vertexTag {
  char* element;    
  char* operation;  
  int isConstant;
  int isVisited;
  struct edgeTag* edge;
  struct vertexTag *next;
} vertex;

typedef struct edgeTag {
    struct vertexTag *connectsTo;
    struct edgeTag *next;
 } edge;

typedef struct graphTag{
  vertex *vertices;
}graph;
\end{lstlisting}

\subsubsection*{Estellung von Abhängigkeitsgraphen}
Zur Erstellung eines Abhängigkeitsgraphen werden Datein mit der Endung "'.depg"' angelegt. Dadurch wird es Möglich eine große Anzahl von Graphen zu erstellen und den Algorithmus auszutesten. Dafür wurde eine Syntax gewählt, die die Knoten, Kanten und Variablen, welche aus einem anderen Block kommen, des Graphen definieren. Dabei ist die Eingabe der Reihenfolge von Wichtigkeit, da ansonsten Zugriffsfehler enstehen können. Es wird verlangt, dass die Knoten zuerst definiert werden und somit die vertikale Struktur beschrieben werden kann. Dabei wird die Definition der Knoten mit \textit{vertex:} eingeleitet und mit einem Komma separiert. Der jeweilige Knoten ist durch Brackets ("'["' , "']"') umschlossen. Hinter der Angabe des Knotennamens folgt die definition des Knotentypes. 
\begin{lstlisting}[caption=Konstruktion eines Abhängigkeitsgraphen., label=list:dg]
vertex: [y(+)],[z(*)],[t1(*)],[t2(-)],[a(const)],[b(const)],[c(const)]
edge:   [y->t1] , [z->t1] , [y->t2] , [z->t2] , [t1->a] , [t1->b] ,[t2->c] 
uevar:  [a],[b],[c],[d]
\end{lstlisting}
Ähnlich werden die Verbindungen der Knoten konstruiert. Eingehend mit dem Akronym \textit{edge:} werden die Kanten mit einem Komma getrennt eingelesen. Eingeklammert in Backets werden zwei Knoten durch einen Pfeil definiert (siehe Listing \ref{list:dg}). 

Jeder Abhängigketsgraph speichert eine Liste von Variablen, die außerhalb des Blockes definiert wurden. Das 
vorangehende \textit{uevar:} leitet eine Liste, die ebenfalls durch Kommas getrennt ist, von Variablen ein. Jede Dekleration wird mit einem Carriage return abgeschlossen.


\chapter{Algorithmus}
\section{Finden von Baumkandidaten}
\section{Blockwiederherstellung in balancierter Form}
%\section{Testing}
\label{Testing}

\subsection{Algorithmus}
\label{Algorithmus}

\subsection{Single- vs. Multi-Core}
\label{SMCore}
\chapter{Fazit}
\ac{THB} ist eine einfache Möglichkeit die Performance von Programmen zu erhöhen. Insbesondere in Anbetracht der Tatsache, dass durch das Fehlen eines Kontrollflusses im Basisblock wenig Analyse nötig ist, ist \ac{THB} bei Multiprozessor-Systemen von großem Nutzen. Andere Systeme mit nur einem Kern bzw. Prozessor können durch diese Optimierungen nicht profitieren. Dies ist der parallelen Ausführung von Operationen geschuldet. Jede Operation muss auf einem eigenen Prozessor/Kern durchgeführt werden. Da heute allerdings meistens Multiprozessor-Systeme genutzt werden, ist die Optimierung eines Programms mit Hilfe des \ac{THB} Algorithmus sehr sinnvoll. Die Kombination mit weiteren Optimierungen bietet die Möglichkeit die Performance noch weiter zu steigern.

Weiterhin haben die Tests, welche zur Kontrolle des Algorithmus durchgeführt wurden, gezeigt, dass er funktioniert und einsetzbar ist. Aktuelle Compiler nutzen ihn schon zur Optimierung. Mit dem Trend zu immer mehr Kernen bzw. Prozessoren wird ein Einsatz dieses Algorithmus auch immer sinnvoller. Programme werden zunehmend komplexer und benötigen immer mehr Rechenleistung. Da die Taktraten mittlerweile an ihre Grenzen geraten, ist nur durch eine Parallelisierung von Programmabläufen eine Verbesserung der Ausführungszeit möglich. Der Einsatz von Optimierungen die an dieser Stelle ansetzen wird somit immer wichtiger.

% Leere Seite nach Inhalt
\newpage 
\thispagestyle{empty}
\quad
\newpage

% Schluss
\pagenumbering{Roman}
\setcounter{page}{\value{last_roman}}
\chapter*{Abkürzungsverzeichnis}
\addcontentsline{toc}{chapter}{Abkürzungsverzeichnis}
% \ac{abk} für automatisch Abkürzung oder Beschreibung
% \acs{abk} für Abkürzung
% \acl{abk} für Beschreibung
% \acf{abk} für Abkürzung + Beschreibung
\begin{acronym}
\setlength{\itemsep}{-\parsep}
\acro{THB}{Tree-Height BalancingL}
\end{acronym}
\newpage
\input{Abbildungsverzeichnis}
\newpage
\phantomsection 
\lstlistoflistings
\addcontentsline{toc}{chapter}{Listings}
\newpage
\phantomsection
\nocite{*}
\printbibliography
\addcontentsline{toc}{chapter}{Literatur}

\end{document}